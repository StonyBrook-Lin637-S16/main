\setcounter{chapter}{-1}
\chapter{Syllabus}
\label{cha:syllabus}
\setcounter{page}{1}
\pagestyle{fancy}

\fcolorbox{gray!25}{gray!25}{%
    \centering
    \begin{tabular}{ll}
        \textbf{Course:} Computational Linguistics 2\quad\qquad\qquad&
        \textbf{Name:} Thomas Graf\\
        \textbf{Course\#:} Lin637 &
        \textbf{Email:} lin637@thomasgraf.net\\
        \textbf{Time:} TR 10:00--11:20am &
        \textbf{Office hours:} Tue 11:30--2:30pm\\
        \textbf{Location:} Psychology A 144 &
        \textbf{Office:} SBS N249\\
        \textbf{Course Website:} tba & %fixme
        \textbf{Personal Website:} \url{http://thomasgraf.net}
    \end{tabular}
}


\section{Course Outline}

\subsection{Contents}

This course serves a specific purpose in our program (see Fig.~\vref{fig:Syllabus_Program}):
it acts as the bridge from introductory courses in linguistics (Syntax 1, Phonology 1, Phonetics) and computational methods (Statistics, Mathematical Methods in Linguistics, Computational Linguistics 1) to advanced courses and seminars in computational\slash mathematical linguistics.
In contrast to the NLP courses offered by the department of computer science, our courses focus on studying the properties of natural language from a computationally informed perspective.
The question is not how computers can solve linguistic tasks, but how language can be conceptualized as a computational problem.
This emphasis is also reflected in the selection of topics for this course.

\begin{itemize}
    \item \textbf{What this course is not about}
        \begin{itemize}
            \item Computers as tools for linguistic research
            \item Programming
            \item Software development for natural language tasks
        \end{itemize}
    \item \textbf{What is not covered but profits from what is covered}
        \begin{itemize}
            \item Speech recognition
            \item OCR
            \item Text generation
            \item Parsing
            \item Semantic analysis
            \item Machine translation
        \end{itemize}
    \item \textbf{List of topics}
        \begin{itemize}
            \item \emph{Phonology}
                \begin{itemize}
                    \item The role of formalization
                    \item String languages
                    \item Subregular hierarchy
                    \item Regular languages
                    \item Generative capacity of phonology
                    \item String transductions
                    \item 2-level morphology
                    \item Equivalence of SPE and OT
                \end{itemize}
            \item \emph{Syntax}
                \begin{itemize}
                    \item Tree languages
                    \item Syntax is more complex than phonology
                    \item Mildly context-sensitive formalisms (TAG, MGs)
                    \item Tree transductions
                    \item Regular representations of MCS formalisms
                    \item Reinterpreting the T-model
                \end{itemize}
        \end{itemize}
\end{itemize}

A detailed outline of the course is given in Tab.~\ref{tab:Syllabus_CourseOutline}.

\begin{table}
    \centering
    \begin{tabular}{rlp{4.5cm}p{4.5cm}}
        \toprule
        \emph{Wk} & \emph{Classes} & \emph{Formal} & \emph{Linguistics} \\
        \toprule
        1         & Jan 27, 29     & What is computation? & Marr's Three Levels\\
        2         & Feb 3, 5       & Formalizing phonology & Why formalize?\\
        3         & Feb 10, 12     & Strictly local languages & Local dependencies\\
        4         & Feb 17, 19     & Subregular hierarchy & How powerful is phonology?\\
        5         & Feb 24, 26     & Regular languages & Abstractness\\
        6         & (DGFS)         & & \\
        7         & Mar 10, 12     & String transductions & SPE-OT equivalence\\
        \midrule
        8         & (Spring Break) & & \\
        \midrule
        9         & Mar 24, 26     & Weak Generative Capacity & $\text{Phonology} < \text{Syntax}$\\
        10        & Mar 31, Apr 2  & Tree languages & Headedness, feature percolation\\
        11        & Apr 7, 9       & Local tree languages & GPSG\\
        12        & (GLOW)         & & \\
        13        & Apr 21, 23     & Recognizable tree languages & GB\\
        14        & Apr 28, 30     & TAG and MGs & Minimalist syntax\\
        15        & May 5, 7       & Tree transductions & Reinterpreting the T-model\\
        \bottomrule
    \end{tabular}
\caption{Tentative course outline}
\label{tab:Syllabus_CourseOutline}
\end{table}

\begin{figure}
    \rotatebox{0}{
        \footnotesize
        \begin{tikzpicture}[
    every node/.style = { draw, thick },
    every path/.style = { ->, thick },
    sug/.style = { dashed },
    req/.style = { },
    ]
    \node (CL2) at (0,0) [align=center] {Computational Linguistics 2\\ (Lin 637)};

    % Prereqs
    \node (Phon) [above=of CL2, xshift=-8em, align=center] {Phonology 1 (Lin 522)\\
                                                                \emph{or}\\
                                                            Phonetics (Lin 523)
                                                        };
    \node (Syntax) [left=of Phon, align=center] {Syntax 1\\ (Lin 521)};
    \node (Math)   [above=of CL2, xshift=8em, align=center] {Statistics (Lin 538)\\
                                                                \emph{or}\\
                                                            Mathematical Methods (Lin 539)
                                                        };
    \node (CL1) [right=of Math, align=center] {CompLing 1\\ (Lin 537)};

    % CS branch
    \node (NLP) [below right=of CL2, xshift= 8em, align=center] {Introduction to NLP\\ (CSE 628)};
    \node (Machine) [below=of NLP, xshift=-8em, align=center]  {Machine Learning\\ (CSE 512)};
    \node (Speech)  [below=of NLP, xshift= 8em, align=center] {Speech Processing\\ (CSE 542)};

    % Linguistics branch
    \node (CompSem) [below=of CL2, xshift=-16em, align=center] {Computational Semantics\\ (Lin 626)};
    \node (CompPhon) [below=of CompSem, align=center] {Computational Phonology\\ (Lin 627)};
    \node (CompSyn) [below=of CompPhon, align=center] {Computational Syntax\\ (Lin 628)};

    \node (Learn) [right=of CompSem, xshift=4em, align=center]  {Learnability\\ (Lin 629)};
    \node (Parse) [right=of CompPhon, xshift=4em, align=center] {Parsing and Processing\\ (Lin 630)};

    % Branches
    \draw[sug] (Syntax) |- (CL2);
    \draw[sug] (Phon) to (CL2);
    \draw[sug] (Math) to (CL2);
    \draw[req] (CL1) |- (CL2);
    \draw[req] (CL1.south -| NLP.north) -- (NLP);
    \draw[sug] (NLP) -| (Machine);
    \draw[sug] (NLP) -| (Speech);
    \draw[sug] (Learn |- CL2.south) -- (Learn); 
    \draw[sug, transform canvas={xshift=2em}] (Parse |- CL2.south) -- (Parse);
    \draw[sug] ($(CL2.south)-(6em,0)$) |- (CompSem);
    \draw[sug] ($(CL2.south)-(5em,0)$) |- (CompPhon);
    \draw[sug] ($(CL2.south)-(4em,0)$) |- (CompSyn);
\end{tikzpicture}

    }
\caption{Computational Linguistics 2 in the curriculum (dashed lines indicate recommendations rather than prerequisites)}
\label{fig:Syllabus_Program}
\end{figure}    

\subsection{Prerequisites}

The only official prerequisite is Computational Linguistics 1 (Lin 537) or comparable programming skills in Python.
It is also helpful to have some basic familiarity with linguistics (phonemes, phrase structure rules, syntactic trees) and mathematics (sets, functions, relations, and propositional logic as covered in Semantics 1, for instance).
You can take an online survey to identify weaknesses, and several introductory readings on these topics are available on the course website.

\section{Teaching Goals}
\begin{itemize}
    \item \textbf{Practical Skills}
        \begin{itemize}
            \item conceptualize a problem in mathematical terms
            \item optimize your programs through the use of adequate algorithms and data structures (dynamic programming techniques, hash tables, etc.)
            \item a more abstract and theoretically informed perspective on current tools and techniques in NLP
            \item an understanding for how linguistic insights can be invoked to simplify NLP tasks
        \end{itemize}
    \item \textbf{Research Skills}
        \begin{itemize}
            \item assess linguistic phenomena from a computational perspective
            \item evaluate linguists' claims about computational efficiency
            \item basic overview of current research in theoretical computational linguistics
            \item use computational concepts to identify new empirical generalizations
            \item bring linguistic data to bear on computational claims
            \item mathematically informed understanding of linguistic theories
        \end{itemize}
\end{itemize}


\section{Grading}
\begin{itemize}
    \item \textbf{Homework}
        \begin{itemize}
            \item weekly exercises, programming assignments, or critical evaluations of assigned readings
            \item a random sample of homeworks will be collected and graded
            \item solutions will be made available online after the due date
            \item Collaboration on homework problems is encouraged as long as you write up the solutions by yourself, using your own words, examples, notation, and code.
        \end{itemize}
        %
    \item \textbf{Readings}
        \begin{itemize}
            \item about two readings per week
            \item you have to collectively write a summary for each reading in the course wiki
        \end{itemize}
        %
    \item \textbf{Survey Squibs}
        \begin{itemize}
            \item By the end of the course, the wiki should contain survey articles on a number of topics not covered in this course (or not covered at the same depth).
            \item You have to pick a topic and write the corresponding survey article.
            \item These articles should be succinct and simple enough that they are comprehensible to a researcher with little exposure to computational linguistics, yet at the same time include enough technical detail that the claims can be verified by somebody with the appropriate background.
                (Why this weird requirement?
                Because that's the recipe for writing a computational paper that can be published in a linguistics journal!)
        \end{itemize}
        %
    \item \textbf{Workload per Credits}
        \begin{itemize}
            \item \emph{3 credits}: homework, readings, squib
            \item \emph{2 credits}: homework, readings
            \item \emph{1 credit}: readings
            \item \emph{0 credits}: none, but I highly recommend that you at least read the assigned papers as they will be important for following the lectures
        \end{itemize}
\end{itemize}


\section{Policies}

\subsection{Contacting me}
\begin{itemize}
    \item Emails should be sent to lin637@thomasgraf.net to make sure they go to my high priority inbox.
        Disregarding this policy means late replies and is a sure-fire way to get on my bad side.
    \item Reply time < 24h in simple cases, possibly more if meddling with bureaucracy is involved.
    \item If you want to come to my office hours and anticipate a longer meeting, please email me so that we can set apart enough time and avoid collisions with other students.
\end{itemize}

\subsection{Special Needs}
If you have any special needs that might impact your class performance (learning disabilities, impaired sight or hearing, etc.), please come to my office hours or contact me via mail so we can make suitable arrangements.
